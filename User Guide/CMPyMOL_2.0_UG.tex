%%%%%%%%%%%%%%%%%%%%%%%%%%%%%%%%%%%%%%%%%
% The Legrand Orange Book
% LaTeX Template
% Version 2.1.1 (14/2/16)
%
% This template has been downloaded from:
% http://www.LaTeXTemplates.com
%
% Original author:
% Mathias Legrand (legrand.mathias@gmail.com) with modifications by:
% Vel (vel@latextemplates.com)
%
% License:
% CC BY-NC-SA 3.0 (http://creativecommons.org/licenses/by-nc-sa/3.0/)
%
% Compiling this template:
% This template uses biber for its bibliography and makeindex for its index.
% When you first open the template, compile it from the command line with the 
% commands below to make sure your LaTeX distribution is configured correctly:
%
% 1) pdflatex main
% 2) makeindex main.idx -s StyleInd.ist
% 3) biber main
% 4) pdflatex main x 2
%
% After this, when you wish to update the bibliography/index use the appropriate
% command above and make sure to compile with pdflatex several times 
% afterwards to propagate your changes to the document.
%
% This template also uses a number of packages which may need to be
% updated to the newest versions for the template to compile. It is strongly
% recommended you update your LaTeX distribution if you have any
% compilation errors.
%
% Important note:
% Chapter heading images should have a 2:1 width:height ratio,
% e.g. 920px width and 460px height.
%
%%%%%%%%%%%%%%%%%%%%%%%%%%%%%%%%%%%%%%%%%

%----------------------------------------------------------------------------------------
%	PACKAGES AND OTHER DOCUMENT CONFIGURATIONS
%----------------------------------------------------------------------------------------

\documentclass[11pt,fleqn]{book} % Default font size and left-justified equations
\usepackage[dvipsnames]{xcolor}
\usepackage{wrapfig}
\usepackage{listings}
\usepackage{textcomp}
\usepackage{smartdiagram}
\usepackage{texshade}
%----------------------------------------------------------------------------------------
\definecolor{dkgreen}{rgb}{0,0.6,0}
\definecolor{gray}{rgb}{0.5,0.5,0.5}
\definecolor{mauve}{rgb}{0.58,0,0.82}
\definecolor{orange}{rgb}{1.0, 0.64, 0}
\definecolor{hotpink}{rgb}{1.0, 0, 0.55}
\definecolor{limegreen}{rgb}{0, 0.95, 0.47}
\definecolor{magenta}{rgb}{0.5, 0, 0.5}
\definecolor{marine}{rgb}{0, 0.26, 0.94}
\definecolor{yellowocre}{rgb}{0.95, 0.79, 0.0}

\lstset{frame=tb,
  language=Python,
  aboveskip=3mm,
  belowskip=3mm,
  showstringspaces=false,
  columns=flexible,
  basicstyle={\small\ttfamily},
  numbers=none,
  numberstyle=\tiny\color{gray},
  keywordstyle=\color{black},
  commentstyle=\color{dkgreen},
  stringstyle=\color{black},
  breaklines=true,
  breakatwhitespace=true,
  tabsize=3
}
%----------------------------------------------------------------------------------------

\input{structure} % Insert the commands.tex file which contains the majority of the structure behind the template

\begin{document}

%----------------------------------------------------------------------------------------
%	TITLE PAGE
%----------------------------------------------------------------------------------------

\begingroup
\thispagestyle{empty}
\begin{tikzpicture}[remember picture,overlay]
\coordinate [below=10.5cm] (midpoint) at (current page.north);
\node at (current page.north west)
{\begin{tikzpicture}[remember picture,overlay]
\node[anchor=north west,inner sep=0pt] at (0,0) {\includegraphics[width=\paperwidth]{background}}; % Background image
\draw[anchor=north] (midpoint) node [fill=ocre!30!white,fill opacity=0.6,text opacity=1,inner sep=1cm]{
\Huge\centering\bfseries\sffamily\parbox[c][][t]{\paperwidth}{\centering \includegraphics[width=0.175\textwidth]{icon.png}\\CMPyMOL 2.0\\[15pt] % Book title
{\Large Contact Map plugin for Python Molecular Viewer}\\[20pt] % Subtitle
{\huge Dr. Venkatramanan Krishnamani}}}; % Author name
\end{tikzpicture}};
\end{tikzpicture}
\vfill
\endgroup

%----------------------------------------------------------------------------------------
%	COPYRIGHT PAGE
%----------------------------------------------------------------------------------------

\newpage
~\vfill
\thispagestyle{empty}

\noindent Copyright \copyright\ 2016 Venkatramanan Krishnamani\\ % Copyright notice

\noindent \textsc{https://github.com/emptyewer/CMPyMOL/releases}\\ % URL

\noindent \textsc{This software is provided under ``The MIT License'' (MIT). \emph{See Section \ref{license}}}\\ % License information

\noindent \textit{First printing, January 2016} % Printing/edition date

%----------------------------------------------------------------------------------------
%	TABLE OF CONTENTS
%----------------------------------------------------------------------------------------

%\usechapterimagefalse % If you don't want to include a chapter image, use this to toggle images off - it can be enabled later with \usechapterimagetrue

\chapterimage{chapter_head_1.pdf} % Table of contents heading image

\pagestyle{empty} % No headers

\tableofcontents % Print the table of contents itself

\cleardoublepage % Forces the first chapter to start on an odd page so it's on the right

\pagestyle{fancy} % Print headers again

%----------------------------------------------------------------------------------------
%	PART
%----------------------------------------------------------------------------------------

\part{Part One}

%----------------------------------------------------------------------------------------
%	CHAPTER 1
%----------------------------------------------------------------------------------------

\chapterimage{chapter_head_2.pdf} % Chapter heading image

\chapter{Installation Instructions}

\section{Pre-compiled Binaries}\index{Installation}
\subsection{Download Link}\label{download link}\index{Download Link}

    Platform-specific compiled binaries (\emph{Mac OS X and Windows}) of \textbf{CMPyMOL} can be downloaded from the below URL. \\

    \texttt{\href{https://github.com/emptyewer/CMPyMOL/releases}{https://github.com/emptyewer/CMPyMOL/releases}}


    \subsection{Prerequisites for pre-compiled binaries}\index{source}

	\begin{lstlisting}
	# 1. Download and Install PyMOL (according to your platform)
	https://www.pymol.org/
	\end{lstlisting}

    \subsection{Mac OS X Compatibility}\index{Mac OS X Compatibility}\label{mac_install}

    \texttt{Mac OS X (10.10+) Yosemite and above}

    \subsection{Windows Compatibility}\index{Windows Compatibility}\label{windows_install}

    \texttt{64-bit or 32 bit Windows 7 and above.\\Note that CMPyMOL itself is a 32-bit software.}

    \subsection{Linux Compatibility}\index{Linux Compatibility}\label{linux_install}

    \texttt{64-bit Linux Binaries not available for this release. The next release will include the linux binaries.}

    \begin{remark}
    Other machine specific binaries will be provided upon request.
    \end{remark}

\clearpage



\section{Running from Source}\index{Running from Source}

CMPyMOL source can be downloaded fromt he following URL.
\begin{lstlisting}
https://github.com/emptyewer/CMPyMOL
\end{lstlisting}

The following libraries are neceassary to run CMPyMOL from source. 

\begin{remark}
All necessary libraries and software dependencies (except PyMOL) are included in the compiled binaries, this requirement is only for running CMPyMOL from source
\end{remark}

\subsection{Pre-requisite Softwares}\index{Pre-requisite Softwares}

\begin{enumerate}
\item \texttt{Python 2.7}
\item \texttt{PyMOL in system \$PATH (for all platforms)}
\item STRIDE secondary structure prediction software properly configured to be in \$PATH
\begin{lstlisting}
http://webclu.bio.wzw.tum.de/stride/
\end{lstlisting}
\end{enumerate}

\begin{remark}
Stride for Mac OSX and Windows is included in the precompiled binary.
\end{remark}


\subsection{Pre-requisite Python Libraries}\index{Pre-requisite Python Libraries}

\begin{enumerate}
\item \texttt{PyQT4}
\item \texttt{matplotlib}
\item \texttt{xmlrpclib}
\item \texttt{numpy}
\end{enumerate}


\section{Open Source License}\index{Open Source License}\label{license}
\begin{lstlisting}

The MIT License (MIT)

Copyright (c) 2016 Venkatramanan Krishnamani

Permission is hereby granted, free of charge, to any person obtaining a copy
of this software and associated documentation files (the "Software"), to deal
in the Software without restriction, including without limitation the rights
to use, copy, modify, merge, publish, distribute, sublicense, and/or sell
copies of the Software, and to permit persons to whom the Software is
furnished to do so, subject to the following conditions:

The above copyright notice and this permission notice shall be included in all
copies or substantial portions of the Software.

THE SOFTWARE IS PROVIDED "AS IS", WITHOUT WARRANTY OF ANY KIND, EXPRESS OR
IMPLIED, INCLUDING BUT NOT LIMITED TO THE WARRANTIES OF MERCHANTABILITY,
FITNESS FOR A PARTICULAR PURPOSE AND NONINFRINGEMENT. IN NO EVENT SHALL THE
AUTHORS OR COPYRIGHT HOLDERS BE LIABLE FOR ANY CLAIM, DAMAGES OR OTHER
LIABILITY, WHETHER IN AN ACTION OF CONTRACT, TORT OR OTHERWISE, ARISING FROM,
OUT OF OR IN CONNECTION WITH THE SOFTWARE OR THE USE OR OTHER DEALINGS IN THE
SOFTWARE.
\end{lstlisting}



\part{Part Two}
%----------------------------------------------------------------------------------------
%	CHAPTER 2
%----------------------------------------------------------------------------------------

\chapter{Basic Usage}\index{CMPyMOL Usage}

Currently both single frame and multi--frame PDB files are supported for analysis by CMPyMOL.

\section{Launch CMPyMOL}\index{Launch CMPyMOL}

CMPyMOL can be launched after installation by clicking on the executable with an icon \includegraphics[scale=0.08]{Pictures/icon.png} (appropriately for Windows and Mac OS X platform). To launch from source simply type \texttt{python CMPyMOL\_2.0.py} in machine appropriate command line shell.

\begin{remark}
If PyMOL executable is not automatically located by CMPyMOL, a pop-up will request the location of PyMOl executable. Select either the location of \texttt{MacPyMOL.app} (for Mac OS X) or \texttt{pymol.exe} (for Windows). The user has to select the location of PyMOL executable only once, after which the path is remembered. In subsequent launching of CMPyMOL if PyMOL is not automatically launched, check the troubleshooting section \ref{troubleshooting}.
\end{remark}

%------------------------------------------------

\section{CMPyMOL Interface}\index{CMPyMOL Interface}

After lanching CMPyMOL, the user will be presented with a window similar to Figure \ref{interface}. This interface provides all the functionalty of the software with no additional pop-ups that distract the user. 

The overall organization of the interactive elements in this interface is such that all overlay's that are displayed on top of contact map are on the right side. The parameters for calculating the contact map and other information are provided on the left. 

All the maps (contact map, heat map of pairwise amino acid contacts and contact density histogram) are plotted as tabs in the central widget. These tabs will be added to the CMPyMOL interface when a PDB file is loaded. The functionatily of each interactive element will be discussed further in the following sections.

\begin{figure}[ht!]
\centering
  \begin{minipage}{\textwidth}
  \centering
      \includegraphics[width=0.9\textwidth]{interface}
      \caption{Main Interface of CMPyMOL.}
  \label{interface}
  \end{minipage}
\end{figure}

% \begin{definition}[Definition name]
% Given a vector space $E$, a norm on $E$ is an application, denoted $||\cdot||$, $E$ in $\mathbb{R}^+=[0,+\infty[$ such that:
% \begin{align}
% & ||\mathbf{x}||=0\ \Rightarrow\ \mathbf{x}=\mathbf{0}\\
% & ||\lambda \mathbf{x}||=|\lambda|\cdot ||\mathbf{x}||\\
% & ||\mathbf{x}+\mathbf{y}||\leq ||\mathbf{x}||+||\mathbf{y}||
% \end{align}
% \end{definition}

\section{Parameters to Calculate Contact Map}\index{Calculate Contact Map}\label{calculate_map_params}

A protein contact map represents the pairwise distance between all possible amino acid residue pairs of a three-dimensional protein structure. These distances are typically calculated between either \texttt{C$\mathrm{\alpha}$ or C$\mathrm{\beta}$} atom. The \texttt{cut-off} distance eliminates all the pairwise distances that are larger than that value. 

The user has the choice to calculate the contact map based on specific parameters (Figure \ref{contactmapcalculate}). The choice of \texttt{C$\mathrm{\alpha}$ or C$\mathrm{\beta}$} atom as the basis atom and the cut-off distance (defaults to 15$\mathrm{\AA}$) for calculating the contact map will be can be customized by the user.

\begin{figure}[ht!]
\centering
  \begin{minipage}[b]{0.35\textwidth}
  \centering
      \includegraphics[width=\textwidth]{parameters}
      \caption{Parameters for Contact Map Calculation}
  \label{contactmapcalculate}
  \end{minipage}
\end{figure}

\begin{remark}
The parameters to calculate contact maps can be set by the user before or after loading the PDB for on-the-fly updates of maps and overlays. 
\end{remark}

\subsection{Load PDB file}\index{Load PDB File}
\begin{figure}[ht!]
\centering
  \begin{minipage}{\textwidth}
  \centering
      \includegraphics[width=0.8\textwidth]{2_choose_pdb}
      \caption{Pop-up dialog for file selection.}
  \label{choosepdb}
  \end{minipage}
\end{figure}

Either a single frame PDB file or a multi-frame PDB from MD simulation trajectory or NMR trajectory can be loaded at this step. The correct format of the PDB file is provided in Section \ref{fileformat} for reference. Other structure formats are currently not supported.

Click on the \includegraphics[width=0.2\textwidth]{Pictures/load_pdb.png} to invoke a pop-up to locate the PDB file. A file dialog (Figure \ref{choosepdb}) will facilitate this choice. Throughout this guide PDB ID: 2MG4 will be used for illustration. 2MG4 is an NMR structure of an artifically designed protein that is a symmetric protein homodimer \cite{mou2015computational}.

After loading the PDB file, there will be two windows presented to the user as shown in Figure \ref{cmpymolwindow} and \ref{pymolwindow}.

\begin{figure}[ht!]
\centering
  \begin{minipage}[b]{0.45\textwidth}
  \centering
      \includegraphics[width=\textwidth]{cmap}
      \caption{CMPyMOL window.}
  \label{cmpymolwindow}
  \end{minipage}
  \quad
  \begin{minipage}[b]{0.45\textwidth}
  \centering
      \includegraphics[width=\textwidth]{pymol}
      \caption{PyMOL window}
  \label{pymolwindow}
  \end{minipage}
\end{figure}

\subsection{PDB Format}\index{PDB Format}\label{fileformat}

\begin{lstlisting}

MODEL X
.
.
.
ATOM ...
ATOM ...
ATOM ...
.
.
.
ENDMDL

\end{lstlisting}

%------------------------------------------------

% \section{Notations}\index{Notations}

% \begin{notation}
% Given an open subset $G$ of $\mathbb{R}^n$, the set of functions $\varphi$ are:
% \begin{enumerate}
% \item Bounded support $G$;
% \item Infinitely differentiable;
% \end{enumerate}
% a vector space is denoted by $\mathcal{D}(G)$. 
% \end{notation}

% %------------------------------------------------

% \section{Remarks}\index{Remarks}

% This is an example of a remark.

% \begin{remark}
% The concepts presented here are now in conventional employment in mathematics. Vector spaces are taken over the field $\mathbb{K}=\mathbb{R}$, however, established properties are easily extended to $\mathbb{K}=\mathbb{C}$.
% \end{remark}

% %------------------------------------------------

% \section{Corollaries}\index{Corollaries}

% This is an example of a corollary.

% \begin{corollary}[Corollary name]
% The concepts presented here are now in conventional employment in mathematics. Vector spaces are taken over the field $\mathbb{K}=\mathbb{R}$, however, established properties are easily extended to $\mathbb{K}=\mathbb{C}$.
% \end{corollary}

% %------------------------------------------------

% \section{Propositions}\index{Propositions}

% This is an example of propositions.

% \subsection{Several equations}\index{Propositions!Several Equations}

% \begin{proposition}[Proposition name]
% It has the properties:
% \begin{align}
% & \big| ||\mathbf{x}|| - ||\mathbf{y}|| \big|\leq || \mathbf{x}- \mathbf{y}||\\
% &  ||\sum_{i=1}^n\mathbf{x}_i||\leq \sum_{i=1}^n||\mathbf{x}_i||\quad\text{where $n$ is a finite integer}
% \end{align}
% \end{proposition}

% \subsection{Single Line}\index{Propositions!Single Line}

% \begin{proposition} 
% Let $f,g\in L^2(G)$; if $\forall \varphi\in\mathcal{D}(G)$, $(f,\varphi)_0=(g,\varphi)_0$ then $f = g$. 
% \end{proposition}

% %------------------------------------------------

% \section{Examples}\index{Examples}

% This is an example of examples.

% \subsection{Equation and Text}\index{Examples!Equation and Text}

% \begin{example}
% Let $G=\{x\in\mathbb{R}^2:|x|<3\}$ and denoted by: $x^0=(1,1)$; consider the function:
% \begin{equation}
% f(x)=\left\{\begin{aligned} & \mathrm{e}^{|x|} & & \text{si $|x-x^0|\leq 1/2$}\\
% & 0 & & \text{si $|x-x^0|> 1/2$}\end{aligned}\right.
% \end{equation}
% The function $f$ has bounded support, we can take $A=\{x\in\mathbb{R}^2:|x-x^0|\leq 1/2+\epsilon\}$ for all $\epsilon\in\intoo{0}{5/2-\sqrt{2}}$.
% \end{example}

% \subsection{Paragraph of Text}\index{Examples!Paragraph of Text}

% \begin{example}[Example name]
% \lipsum[2]
% \end{example}

% %------------------------------------------------

% \section{Exercises}\index{Exercises}

% This is an example of an exercise.

% \begin{exercise}
% This is a good place to ask a question to test learning progress or further cement ideas into students' minds.
% \end{exercise}

% %------------------------------------------------

% \section{Problems}\index{Problems}

% \begin{problem}
% What is the average airspeed velocity of an unladen swallow?
% \end{problem}

% %------------------------------------------------

% \section{Vocabulary}\index{Vocabulary}

% Define a word to improve a students' vocabulary.

% \begin{vocabulary}[Word]
% Definition of word.
% \end{vocabulary}

%----------------------------------------------------------------------------------------
%	PART
%----------------------------------------------------------------------------------------

\part{Part Two}

%----------------------------------------------------------------------------------------
%	CHAPTER 3
%----------------------------------------------------------------------------------------

\chapterimage{chapter_head_1.pdf} % Chapter heading image

\chapter{Functionality}\index{CMPyMOL Functionality}

The overlays in CMPyMOL enriches the interpretation of contact maps.

\section{Contact Map}\index{Contact Map}

\begin{figure}[ht!]
\centering
  \begin{minipage}{\textwidth}
  \centering
      \includegraphics[width=1.1\textwidth]{cmpymol-annotated}
      \caption{Main CMPyMOL Window and Its Functionality}
  \label{cmpymol_annot}
  \end{minipage}
\end{figure}

The main window of CMPyMOL (Figure \ref{cmpymol_annot}) provides controls for all the selection, overlay and plots to analyze contact maps. The overlays (the toggle buttons on the right of the contact map) superpose chemical and structural information on top of the contact map when activated. The plots (buttons on the left side of the contact map) pops open a new window that provides an overview of the nature of contacts.

\subsection{Variance Contact Map}

\begin{figure}[ht!]
\centering
  \begin{minipage}[b]{0.45\textwidth}
  \centering
      \includegraphics[width=\textwidth]{variance_cmpymol}
      \caption{Variance Contact Map. Showing the regions of highest flexibility.}
  \label{variance1}
  \end{minipage}
  \quad
  \begin{minipage}[b]{0.45\textwidth}
  \centering
      \includegraphics[width=\textwidth]{variance_pymol}
      \caption{Selecting a high variance region in the intra-subunit quadrant reveals in the PyMOL window that the region is a flexible loop.}
  \label{variance2}
  \end{minipage}
\end{figure}

Variance map can be activated by clicking on the button \includegraphics[width=0.2\textwidth]{variance_btn}. The {\color{red} red} rectangles in figure \ref{variance1} overlays the secondary structure information of the PDB (more in section \ref{secondary_structure}).

The PDB loaded as an example here is a NMR structure with multiple-frames (10 models). When a multi-frame PDB is loaded, the ``Variance Contact Map'' button is enabled and it allows for calculating the variance in contact distance for a residue pair. Using \includegraphics[width=0.2\textwidth]{frames_btns} buttons the user can progress through the different models. Such frame switch will update all the maps in CMPyMOL along with updating the model itself in the PyMOL window.

This allows for readily identifying flexible regions of the protein. In figure \ref{variance1} a selection highlights the intra-protein flexible region with high variance. In the PyMOL window it can be verified that indeed the region of high variance is on a loop (Figure \ref{variance2}).

\section{Selections}\index{Selecting Residue Ranges}\label{selections}

Selecting a particular region by clicking-n-dragging on a portion of the CMPyMOL generated contact map (shown as a {\color{magenta} magenta} selection box Figure \ref{cmpymolwindow_selection}) highlights the corresponding region in the PyMOL window (Figure \ref{pymolwindow_selection}). Since contact map is a pairwise interaction matrix, the two interacting regions are colored {\color{hotpink} hotpink} and {\color{limegreen} limegreen}. The atom based on which the distances for the contact map are highlighted as ``spheres''.

\begin{figure}[ht!]
\centering
  \begin{minipage}[b]{0.45\textwidth}
  \centering
      \includegraphics[width=\textwidth]{selection_cmpymol}
      \caption{Selection in CMPyMOL window.}
  \label{cmpymolwindow_selection}
  \end{minipage}
  \quad
  \begin{minipage}[b]{0.45\textwidth}
  \centering
      \includegraphics[width=\textwidth]{selection_pymol}
      \caption{Focus on the selected region in PyMOL window.}
  \label{pymolwindow_selection}
  \end{minipage}
\end{figure}

\section{Overlays}

Overlays in CMPyMOL provide a intuitive superposing of chemical and structural information on top of the contact map. This allows of correlating interacting residues with its type (chemical or structural).

\subsection{Secondary Structure}\label{secondary_structure}\index{Secondary Structure}

The button \includegraphics[width=0.2\textwidth]{Pictures/ss_btn.png} toggles the overlaying the secondary structure of protein onto the contact map (Figure \ref{ssoverlay}).

The secondary structure overlay superposes $\alpha$-helical and $\beta$-sheet as {\color{red} red} and {\color{green} green} translucent rectangles, respectively. Notedly, the selection from the last section (Section \ref{selections}) is in a $\alpha$-helix -- $\alpha$-helix interaction, which can be confirmed by the selection highlighted in PyMOL window (Figure \ref{pymolwindow_selection}).

\begin{figure}[ht!]
\centering
  \begin{minipage}{\textwidth}
  \centering
      \includegraphics[width=0.75\textwidth]{ss_cmpymol}
      \caption{Secondary structure overlay on top of contact map.}
  \label{ssoverlay}
  \end{minipage}
\end{figure}

\subsection{Charged Interactions}\index{Charged Interactions}

Similar to secondary structure overlay, by clicking on the charged interaction toggle button \includegraphics[width=0.2\textwidth]{Pictures/charge_btn.png} draws a overlay that highlights contact points where two charged residues are located within the cut-off distance specified in Section \ref{calculate_map_params} (Figure \ref{chargedoverlay}). These charged interaction points are highlighted as {\color{marine} marine} pixels on top of the contact map.

\begin{figure}[ht!]
\centering
  \begin{minipage}{\textwidth}
  \centering
      \includegraphics[width=0.75\textwidth]{charged_cmpymol}
      \caption{Charge-charge interactions highlighted on top of contact map.}
  \label{chargedoverlay}
  \end{minipage}
\end{figure}

\subsection{Hydrophobic Interactions}\index{Hydrophobic Interactions}

Hydrophobic interactions overlay can be toggled by clicking the button \includegraphics[width=0.2\textwidth]{Pictures/hydro_btn.png}. This highlights the points of contact map where two hydrophobic residues are coming in contact. These interaction points are highlighted as {\color{yellowocre} yellow} over the contact map (Figure \ref{hydrophobicoverlay}).

\begin{figure}[ht!]
\centering
  \begin{minipage}{\textwidth}
  \centering
      \includegraphics[width=0.75\textwidth]{hydrophobic_cmpymol}
      \caption{Hydrophobic interactions highlighted on top of contact map.}
  \label{hydrophobicoverlay}
  \end{minipage}
\end{figure}

\subsection{B-factor Overlay}\index{Debye-Waller Factor}

In a similar fashion B-factor (if those values are listed in the PDB) highlights the points with a larger value than set with the slider below the B-factor button (in the case shown in Figure \ref{bfactor_params}, the highlighted contact points have a b-factor larger than 25). These interaction points are highlighted as {\color{cyan} cyan} over the contact map.

\begin{figure}[ht!]
\centering
  \begin{minipage}{\textwidth}
  \centering
      \includegraphics[width=0.35\textwidth]{bfactor_params}
      \caption{Parameters to draw an overlay to display B-factor. The contact points with b-factor larger than cutoff (in brackets) is colored in {\color{cyan} cyan}.}
  \label{bfactor_params}
  \end{minipage}
\end{figure}

\subsection{User--defined Pairwise Aminoacid Interactions}\index{Pairwise Amino Acid Selections}


\begin{figure}[ht!]
\centering
  \begin{minipage}[b]{0.45\textwidth}
  \centering
      \includegraphics[width=\textwidth]{pairwise_cmpymol}
      \caption{User selected interaction between Glutamic Acid and Lysine residues with a a cutoff distance of 15$\mathrm{\AA}$.}
  \label{userdefinedoverlay}
  \end{minipage}
  \quad
  \begin{minipage}[b]{0.45\textwidth}
  \centering
      \includegraphics[width=\textwidth]{selection_pymol}
      \caption{The residues corresponding with the selection (Figure \ref{userdefinedoverlay}) are highlighted in the PyMOL window.}
  \label{userdefinedoverlay_pymol}
  \end{minipage}
\end{figure}

In addition to the above overlays, the user can choose to highlight interaction points of two specific aminoacids. In Figure \ref{userdefinedoverlay}, the interactions of Lysine and Glutamic acid residues are highlited in the PyMOL window (Figure \ref{userdefinedoverlay_pymol}).

\section{Plots}\index{Plots}

Plots in CMPyMOL provide statistical analysis of the contact points.

\subsection{Pairwise Heatmap}\index{Pairwise Heatmap}

\begin{figure}[ht!]
\centering
  \begin{minipage}{\textwidth}
  \centering
      \includegraphics[width=0.75\textwidth]{heatmap_cmpymol}
      \caption{Heatmap of pairwise residue-residue interaction map. The residues are listed according to increasing hydrophobicity}
  \label{heatmap}
  \end{minipage}
\end{figure}

This heatmap counts the number of pairwise contacts of a given aminoacid to the rest of the other aminoacids (Figure \ref{heatmap}). The order of aminoacids in this plot are arranged according to their hydrophobicity. The color scale shows the number of each pairwise contacts in the protein.
The plot is interactive and clicking on any ``box'' will correspondingly update the pymol window corresponding to the selection.

\subsection{Contact Density Histogram}\index{Contacts Density Histogram}

\begin{figure}[ht!]
\centering
  \begin{minipage}{\textwidth}
  \centering
      \includegraphics[width=0.75\textwidth]{contact_density_cmpymol}
      \caption{The residue-wise density of contacts in the PDB.}
  \label{contactdensity}
  \end{minipage}
\end{figure}

The contact histogram plot, graphs the density of contacts with respect to residue position (Figure \ref{contactdensity}). This plot is interactive and clicking on a particular bar selects the corresponding residues and the surrounding contacts in the PyMOL window.

% \section{Table}\index{Table}

% \begin{table}[h]
% \centering
% \begin{tabular}{l l l}
% \toprule
% \textbf{Treatments} & \textbf{Response 1} & \textbf{Response 2}\\
% \midrule
% Treatment 1 & 0.0003262 & 0.562 \\
% Treatment 2 & 0.0015681 & 0.910 \\
% Treatment 3 & 0.0009271 & 0.296 \\
% \bottomrule
% \end{tabular}
% \caption{Table caption}
% \end{table}

% %------------------------------------------------

% \section{Figure}\index{Figure}

% \begin{figure}[h]
% \centering\includegraphics[scale=0.5]{placeholder}
% \caption{Figure caption}
% \end{figure}

%----------------------------------------------------------------------------------------
%	BIBLIOGRAPHY
%----------------------------------------------------------------------------------------

\chapter*{Bibliography}
\addcontentsline{toc}{chapter}{\textcolor{ocre}{Bibliography}}
\section*{Books}
\addcontentsline{toc}{section}{Books}
\printbibliography[heading=bibempty,type=book]
\section*{Articles}
\addcontentsline{toc}{section}{Articles}
\printbibliography[heading=bibempty,type=article]

%----------------------------------------------------------------------------------------
%	INDEX
%----------------------------------------------------------------------------------------

\cleardoublepage
\phantomsection
\setlength{\columnsep}{0.75cm}
\addcontentsline{toc}{chapter}{\textcolor{ocre}{Index}}
\printindex

%----------------------------------------------------------------------------------------

\end{document}
